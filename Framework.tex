% !TEX TS-program = pdflatex
% !TEX encoding = UTF-8 Unicode

% This is a simple template for a LaTeX document using the "article" class.
% See "book", "report", "letter" for other types of document.

\documentclass[10pt]{article} % use larger type; default would be 10pt

\usepackage[utf8]{inputenc} % set input encoding (not needed with XeLaTeX)
\PassOptionsToPackage{hyphens}{url}\usepackage{hyperref}
\usepackage{lipsum}
\usepackage{hyperref}
\usepackage{graphicx}

%%% Examples of Article customizations
% These packages are optional, depending whether you want the features they provide.
% See the LaTeX Companion or other references for full information.

%%% PAGE DIMENSIONS
\usepackage{geometry} % to change the page dimensions
\geometry{a4paper} % or letterpaper (US) or a5paper or....
% \geometry{margin=2in} % for example, change the margins to 2 inches all round
% \geometry{landscape} % set up the page for landscape
%   read geometry.pdf for detailed page layout information

\usepackage{graphicx} % support the \includegraphics command and options
\usepackage{bbding}


% \usepackage[parfill]{parskip} % Activate to begin paragraphs with an empty line rather than an indent

%%% PACKAGES
\usepackage{booktabs} % for much better looking tables
\usepackage{array} % for better arrays (eg matrices) in maths
\usepackage{paralist} % very flexible & customisable lists (eg. enumerate/itemize, etc.)
\usepackage{verbatim} % adds environment for commenting out blocks of text & for better verbatim
\usepackage{subfig} % make it possible to include more than one captioned figure/table in a single float
% These packages are all incorporated in the memoir class to one degree or another...
\usepackage[retainorgcmds]{IEEEtrantools}


%%% HEADERS & FOOTERS
\usepackage{fancyhdr} % This should be set AFTER setting up the page geometry
\pagestyle{fancy} % options: empty , plain , fancy
\renewcommand{\headrulewidth}{0pt} % customise the layout...
\lhead{}\chead{}\rhead{}
\lfoot{}\cfoot{\thepage}\rfoot{MPA / 2017-04-10\\Restricted}

%%% SECTION TITLE APPEARANCE
\usepackage{sectsty}
\allsectionsfont{\sffamily\mdseries\upshape} % (See the fntguide.pdf for font help)
% (This matches ConTeXt defaults)

%%% ToC (table of contents) APPEARANCE
\usepackage[nottoc,notlof,notlot]{tocbibind} % Put the bibliography in the ToC
\usepackage[titles,subfigure]{tocloft} % Alter the style of the Table of Contents
\renewcommand{\cftsecfont}{\rmfamily\mdseries\upshape}
\renewcommand{\cftsecpagefont}{\rmfamily\mdseries\upshape} % No bold!

%%% END Article customizations

%%% The "real" document content comes below...

\title{Data analytics framework}
\author{Magnus Palm}
\date{2017-04-10} % Activate to display a given date or no date (if empty),
         % otherwise the current date is printed 

\begin{document}
\maketitle

\section*{Introduction}
Companies that inject data and analytics into their operations show productivity rates and profitability that are 5\% to 6\% higher than those of their peers \footnote{See Dominic Barton and David Court, `Making advanced analytics work for you.`\textit{Harvard Business Review,} October 2012, Volume 90, Number 10, pp. 78-83.}.

Three mutually supportive capabilities must be in place to exploit data and analytics. An analytics group must have the ability to, identify, recombining, and manage multiple sources of data. An example is the ability to combine sales records with running hours of customer's installed base, and visualizing how this adds value. Second, the ability to build analytics models for translating results, predicting and optimizing outcomes, which results in actionable insights from the data. Last, the organization must have the management strength to transform the organization so that the data and models actually yield better decisions. An example here is to enforce usage of the SOS\footnote{Service Opportunity Sizing} framework in creating the cluster Business plans.

The following features serve as a foundation for the activities, which this document will elaborate around.


\begin{enumerate}
\item A clear strategy for how to use data and analytics to compete
\item Deployment of the right technology architecture and capabilities
\end{enumerate}


\tableofcontents
\listoffigures
\listoftables

%------------------------------------------------
\section{Executive summary}

There is a continuous shift in many businesses towards becoming more data-driven. Companies that understand the value that is generated in their own internal databases combined with external data sets will have a competitive advantage. The rest will have to accept being acted upon in the market rather than being in control of the first-mover advantage.

Tetra Pak sits on unique Master Data and domain knowledge. This is data that is not public and also not easily collected and analyzed by competing asset care companies. The analytics infrastructure is open for anyone to tap into, it is publicly available through for example Microsoft Azure cloud solutions. Our sustainable competitive advantage in the market place is created by using such infrastructure, applying analytical skills on the data available as well as collecting other data where needed, in such fashion as adding value to our customers. 

The roadmap starts with identifying the most critical business decisions that must be taken, and the cost of advanced data analytics that need to be applied at delivering data-driven and fact based decision support.

\subsection{Recommendation}

It has been proven over the last few years that business data presented in various ways have had a positive impact on how we act on facts and numbers, as for example the roll-out of SOS in GME\&A that shows a hockey-stick development of SOS generated spare part sales in some markets. The recommendation is to continue analyzing data based on critical business decisions that need to be taken annually. However, the lack of resources in supplying this data, and the unstructured scattered work that is being done globally in the markets today, is unproductive and not optimized. Rather a dedicated team, preferably in the proximity of Lund should supply and lead this service.

%------------------------------------------------
\section{Situational analysis}
\subsection{PEST}
Here follows an analysis aiming at identifying "Big Picture" Opportunities and Threats in Data Analytics in our business. 

The analysis is divided into the four areas: Political, Economic, Socio-Cultural, and Technological.

\subsubsection{Political}
Governments are likely to impose more regulations around data collections, privacy concerns and possibly on data transfers between countries e.g., U.S. Export control.
An example that illustrates how we continuously adapt to legislation is the data collection and analysis around Service work utilization. We cannot drill down in the analysis to the lowest level as that would link time utilization to individual FSEs.

The risk of political instability in markets is mitigated by subscribing to external partners that scans the news-feeds for early warning signs. Already today the company subscribes to a Business Travel Safety service from an external company that most likely scans live news-feeds from for example Reuters. In the past companies scanned newspapers for news about political deregulation in their business field. Changes in regulations in the food industry that creates new business opportunities will continue. However, today such opportunities are often found using data analytics on live news data.

\subsubsection{Economic}
There is a continuous shift in many businesses towards becoming more data-driven. Companies that understand the value that is generated in their own internal databases combined with external data sets will have a competitive advantage. The rest will have to accept being acted upon in the market rather than being in control of the first-mover advantage.

\subsubsection{Socio-Cultural}
Data-driven companies understand the value in collecting data, internal data as well as external data from customers and consumers. Google is a good example of a company that both collects data and give it back to consumers as value added data. The benefit for the consumer is for example a better dining experience as you can tap into restaurant reviews in google maps. As long as privacy concerns are managed, more consumers tend to accept the sharing and collection of data.

Likewise, companies in B2B transactions become increasingly aware of the risks but also the benefits in sharing and collecting data if they see the return in added value services. In our business we have the example of TPMS OnLine where the collection of running hours and service event logging, generates work orders through an Asset Care Center. The customers benefit from higher Equipment availability provided that they share information and that Tetra Pak returns value added services. 

In the future customers will, by being influenced by the Socio-Cultural environment, not only accept data utilization, but demand that they receive data-driven services in the entire value chain. Marketing, Production process optimizations, and Asset Care are areas in the value chain that hold opportunities for such value added transactions.

As consumers we also expect innovative new products and services through recombining data and technology applications, as for example self-driving cars. Industrial customers expect similar fusions of creativity and technology applications to take place.

\subsubsection{Technological}
In the technology area we see many developments taking place, largely driven by the infrastructure development around tech giants such as as Google. It is important to understand that the software and infrastructure development is driven by free and community driven developments as that has proven to have many benefits over proprietary software developments. Among the drivers are the lower development costs and speed and diversity of software services that are born this way. In short, many of the most powerful software solutions and cloud based server infrastructures are available to anyone that cares to learn, contribute to, and use them.

Developments take place in the areas of:
\begin{itemize}
				\item Extraction, Transformation, and Loading (ETL) of data from many sources, and in more or less real time
				\item Batch processing of large data sets that consist of unstructured data\footnote{Batch processing of large data sets is effectively done in tools such as Hadoop 'http://hadoop.apache.org' or Hive 'http://hive.apache.org'}
				\item Databases that are more flexible in modeling data as data structures change and business needs change \footnote{Traditional databases are SQL based, during the last 10 years many other types of databases were developed in the NoSQL field.}
				\item Data analytics \footnote{Powerful platforms for data analytics are available from the academical and research fields such as for example Anaconda 'https://www.anaconda.com/'}
				\item Data visualizations
\end{itemize}

Large corporate organizations that strive to leverage on the technology development in this field need to find a balance between standardization of IT tools and IT security, and on the other hand, allow data analytics team to access and use the latest technology to drive innovation as well as attract and retain talent in this area.

%------------------------------------------------
\section{Strategy}
The strategy outlines how we will reach our vision, hence I will make an assumption of a vision that goes along the line \textit{TS is data-driven in allocating resources to customer focused activities, thereby ensuring a sustainable competitive advantage in the market}.

\subsection{Scope}

Data analytics is part of:
\begin{itemize}
			\item Product development of new value added data-driven products that increase availability of the customer's capital assets, optimize the customer's production process to drive down cost, and enable a world class experience in service and parts logistics 	
			\item Business analytics of data in the OFSP process, including, but not limited to, Master Data, and Sales 
			\item  Transformation of TS to a Data-Driven organization that is able and motivated to act on insights from data 
\end{itemize}

\subsection{Goals and Objectives}

To reach the vision following must happen:
\begin{itemize}
				\item Recruit talent in Data Science and form a dedicated team of 2-3 people with Business domain experience (6 months)
				\item Collect needs in data analytics from Market Companies, the Customer's Voice, Product owners, and TS (3 months)
				\item Quantify ROI and Sales targets (at time of Budget setting)
				\item Deliver one (1) Success case in increasing sales in a spare part category with proven quantified reduction of customer's production downtime translated to absolute monetary value (9 months)
				\item Change management program that transformed one (1) Market company's sales team in becoming data-driven (18 months), thereafter global deployment of change management program
\end{itemize}

\subsection{Resource deployments}

We decide on which products we deploy resources to by Market research and Technology scouting as to identifying needs and emerging trends that will enable the creation of innovative and value added products.

\subsection{Sustainable competitive advantage}

Tetra Pak sits on unique Master Data and domain knowledge. This is data that is not public and also not easily collected and analyzed by competing asset care companies. The analytics infrastructure is open for anyone to tap into, it is publicly available through for example Microsoft Azure cloud solutions. Our sustainable competitive advantage in the market place is created by using such infrastructure, applying analytical skills on the data available as well as collecting other data where needed, in such fashion as adding value to our customers. 

Creating products based on data is not different from creating other products in the sense that one needs creativity, market research and a vision of what the solutions shall do for the user. Data based products are also not only fulfilling one primary need, they are also providing secondary experiences consumed through for example well designed apps or e-Business Portals.

Marketing and selling data based products are different from the traditional reactive sales of high margin Spare Parts, and will require sales training and other channels to the market.

\subsection{Synergies}

Synergies and economies of scale can be achieved by extending the scope of data-driven business insights to other parts of the organization such as, but not limited to, TS Packaging and Customer Management. Synergies can be achieved simply by merging data sources from TS Packaging and Processing in serving integrated customer accounts such as Coca-Cola. Sales meta data of Packaging material, Running hours, and M\&SP will bring synergies in new data insights.


%------------------------------------------------
\section{Execution}

The rational in making a plan is in being able to envision a path to the fulfillment of the strategy and to measure progress against goals in relation to the resources employed. Various scenarios can be handled by means of a well designed and flexible plan, which a steering group or management team can track and see when and where resources will be needed.


\subsection{Developing a Data-and-Analytics Plan}

To be successful in the organization, the investment in building up a data-driven TS business must start delivering results in 6-9 months. This will keep momentum going and ensure continued organizational support and focus. The first step is to decide what the outcome should be, e.g., increase sales of pistons by \textit{x} \% in market \textit{y} after \textit{z} months and thereby breaking a declining sales trend. The ROI is now quantifiable and a business case can be developed. It is important to set goals\footnote{SMART, i.e, Specific, Measurable, Achievable, Resource based, and Time bound} before starting data mining activities ad hoc. With a set of goals the strategy to get there can be developed. Obviously, deep dives into data will not in itself increase sales in the example above. However, data-driven decisions that correctly allocate scarce sales resources will. This also implies that the sales organization in the example currently have an insufficient decision making model for resource deployment, and is sufficiently approachable and open for changing to data-driven decision models.


\subsection{Roadmap}

Tempting as it may be - to start mining data and extract insights and hope for the best, is not a sustainable way forward. Rather the roadmap starts with identifying the most critical business decisions that must be taken, and the cost of advanced data analytics that need to be applied at delivering data-driven and fact based decision support. This is a cost-benefit analysis. The generated value vs the price tag.

Start with the Decision Making Criteria and from there derive the Data Definitions. The next step in the roadmap is to shortlist the top 20\% of the critical business decisions that will impact 80\% of the value for the business over the next business cycle, and or business outcomes. Using this shortlist it is possible to define the data needed, the data sources, and what data insights to deliver to support the decisions.

Hereafter, we establish gaps in data availability, quality, granularity\footnote{Data granularity is how often the data is gathered, e.g., Daily vs Monthly reported sales data}, and historical period. Connected to this is also how the data shall be presented, in what form, drillable interactive Dashboards, and what analytics are needed to support the consuming of insights by the user. It is of course also necessary to know the audience, is it a management team, and therefore data literate, or is it data to be used in supporting an important customer sales meeting?

Some decisions are one-off as for example investment in new storage of Spare Parts, other decisions are annually recurring and as such benefit from an automated data collection and presentation system as for example a Dashboard.

\subsection{Data access}

In general, Tetra Pak has very good access to internal data via SAP and other databases that we subscribe to. We also have large data sets being collected from customer's Installed Base (IB), however additional data is often hard to access. The quality and consistency of data is also a challenge that needs to be overcome. Some of the more valuable information in this area is related to actual running hours, type of product and its effect on Wear\&Tear of spare parts, Build specifications of IB, and tracking of spare parts to IB.

\subsection{S/W Tools}

There are several software solutions that an analytics team need access to, and will have access to, as these are not necessarily expensive at all. They range from Microsoft Excel, rapid prototyping tools such as R and Python, to Server Client build software C\#, and Java. The nature of software languages are such that Data Scientists and Developers tend to become proficient in a few of them, and to attract talent it is important to have some flexibility in what tools will be allowed from a corporate perspective. More important is that developed solutions are well documented.

\subsection{HR Profiles}

The team working with data analytics needs a wide spectrum of skills and strength. 

People skills and Presentation skills are needed in selling the benefits of data-driven decisions and actions, as well as when collecting needs. People in general do not like o be told by a software to make directed and traceable cold sales calls, -this needs to be sold in.

Strong analytical skills and domain knowledge is needed from the person doing the analysis. Very large complex data set mergers need help from a Data Scientist to utilize the server and database design in the most efficient way\footnote{TPMS OnLine is now experiencing a bottleneck in how many more assets can be connected before either a redesign of the architecture is needed, adding more memory, or alternatively redesigning the load algorithms.}.


%------------------------------------------------
\section{Summary}

There is a continuous shift in many businesses towards becoming more data-driven. Companies that understand the value that is generated in their own internal databases combined with external data sets will have a competitive advantage. The rest will have to accept being acted upon in the market rather than being in control of the first-mover advantage.

Companies in B2B transactions become increasingly aware of the risks but also the benefits in sharing and collecting data if they see the return in added value services.

In the technology area we see many developments taking place, largely driven by the infrastructure development around tech giants such as as Google.


Many of the most powerful software solutions and cloud based server infrastructures are available to anyone that cares to learn, contribute to, and use them.


Three mutually supportive capabilities must be in place to exploit data and analytics. An analytics group must have the ability to, identify, recombining, and manage multiple sources of data. An example is the ability to combine sales records with running hours of customer's installed base, and visualizing how this adds value. Second, the ability to build analytics models for translating results, predicting and optimizing outcomes, which results in actionable insights from the data. Last, the organization must have the management strength to transform the organization so that the data and models actually yield better decisions.


Tetra Pak sits on unique Master Data and domain knowledge. This is data that is not public and also not easily collected and analyzed by competing asset care companies. The analytics infrastructure is open for anyone to tap into, it is publicly available through for example Microsoft Azure cloud solutions. Our sustainable competitive advantage in the market place is created by using such infrastructure, applying analytical skills on the data available as well as collecting other data where needed, in such fashion as adding value to our customers. 

The roadmap starts with identifying the most critical business decisions that must be taken, and the cost of advanced data analytics that need to be applied at delivering data-driven and fact based decision support.

%-----------------------------------------------
\section{Recommendations}

The future is clearly geared towards utilizing the data gathered in internal business transaction systems and externally available data. However, far from all corporations are able to increase productivity and sales based on data analytics investments. Consequently, a first step is to make a prestudy of the costs and benefits, and leading up to a business case. The feasibility in doing this organizationally must also be taken into consideration. Teaming up with TS Packaging would give more leverage on resources, and having resources based in the geographical proximity of Sweden.

It has been proven over the last few years that business data presented in various ways have had a positive impact on how we act on facts and numbers, as for example the roll-out of SOS in GME\&A that shows a hockey-stick development of SOS generated spare part sales in some markets. The recommendation is to continue analyzing data based on critical business decisions that need to be taken annually. However, the lack of resources in supplying this data, and the unstructured scattered work that is being done globally in the markets today, is unproductive and not optimized. Rather a dedicated team, preferably in the proximity of Lund should supply and lead this service.

A list of future products and services based on advanced analytics has on purpose been left out from the report as the recommendation is to start in listing the top critical business decisions the company takes annually and their supporting data-insights rather than starting the other way around. However, a brainstorming session could without too much effort probably list several exiting ideas of what we could offer our customers in the future.
%------------------------------------------------

\end{document}
