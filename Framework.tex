% !TEX TS-program = pdflatex
% !TEX encoding = UTF-8 Unicode

% This is a simple template for a LaTeX document using the "article" class.
% See "book", "report", "letter" for other types of document.

\documentclass[10pt]{article} % use larger type; default would be 10pt

\usepackage[utf8]{inputenc} % set input encoding (not needed with XeLaTeX)
\PassOptionsToPackage{hyphens}{url}\usepackage{hyperref}
\usepackage{lipsum}
\usepackage{hyperref}
\usepackage{graphicx}

%%% Examples of Article customizations
% These packages are optional, depending whether you want the features they provide.
% See the LaTeX Companion or other references for full information.

%%% PAGE DIMENSIONS
\usepackage{geometry} % to change the page dimensions
\geometry{a4paper} % or letterpaper (US) or a5paper or....
% \geometry{margin=2in} % for example, change the margins to 2 inches all round
% \geometry{landscape} % set up the page for landscape
%   read geometry.pdf for detailed page layout information

\usepackage{graphicx} % support the \includegraphics command and options
\usepackage{bbding}


% \usepackage[parfill]{parskip} % Activate to begin paragraphs with an empty line rather than an indent

%%% PACKAGES
\usepackage{booktabs} % for much better looking tables
\usepackage{array} % for better arrays (eg matrices) in maths
\usepackage{paralist} % very flexible & customisable lists (eg. enumerate/itemize, etc.)
\usepackage{verbatim} % adds environment for commenting out blocks of text & for better verbatim
\usepackage{subfig} % make it possible to include more than one captioned figure/table in a single float
% These packages are all incorporated in the memoir class to one degree or another...
\usepackage[retainorgcmds]{IEEEtrantools}


%%% HEADERS & FOOTERS
\usepackage{fancyhdr} % This should be set AFTER setting up the page geometry
\pagestyle{fancy} % options: empty , plain , fancy
\renewcommand{\headrulewidth}{0pt} % customise the layout...
\lhead{}\chead{}\rhead{}
\lfoot{}\cfoot{\thepage}\rfoot{MPA / 2017-04-10\\Restricted}

%%% SECTION TITLE APPEARANCE
\usepackage{sectsty}
\allsectionsfont{\sffamily\mdseries\upshape} % (See the fntguide.pdf for font help)
% (This matches ConTeXt defaults)

%%% ToC (table of contents) APPEARANCE
\usepackage[nottoc,notlof,notlot]{tocbibind} % Put the bibliography in the ToC
\usepackage[titles,subfigure]{tocloft} % Alter the style of the Table of Contents
\renewcommand{\cftsecfont}{\rmfamily\mdseries\upshape}
\renewcommand{\cftsecpagefont}{\rmfamily\mdseries\upshape} % No bold!

%%% END Article customizations

%%% The "real" document content comes below...

\title{Data analytics framework}
\author{Magnus Palm}
\date{2017-04-10} % Activate to display a given date or no date (if empty),
         % otherwise the current date is printed 

\begin{document}
\maketitle

\section*{Introduction}
Companies that inject data and analytics into their operations show productivity rates and profitability that are 5\% to 6\% higher than those of their peers \footnote{See Dominic Barton and David Court, `Making advanced analytics work for you.`\textit{Harvard Business Review,} October 2012, Volume 90, Number 10, pp. 78-83.}.

Three mutually supportive capabilities must be in place to exploit data and analytics. An analytics group must have the ability to, identify, recombining, and manage multiple sources of data. An example is the ability to combine sales records with running hours of customer's installed base, and visualizing how this adds value. Second, the ability to build analytics models for translating results, predicting and optimizing outcomes, which results in actionable insights from the data. Last, the organization must have the management strength to transform the organization so that the data and models actually yield better decisions. An example here is to enforce usage of the SOS\footnote{Service Opportunity Sizing} framework in creating the cluster Business plans.

The following features serve as a foundation for the activities, which this document will elaborate around.


\begin{enumerate}
\item A clear strategy for how to use data and analytics to compete
\item Deployment of the right technology architecture and capabilities
\end{enumerate}


\tableofcontents
\listoffigures
\listoftables

%------------------------------------------------
\section{Executive summary}
text

\subsection{Recommendation}

text

%------------------------------------------------
\section{Situational analysis}
\subsection{PEST}
Here follows an analysis aiming at identifying "Big Picture" Opportunities and Threats in Data Analytics in our business. 

The analysis is divided into the four areas: Political, Economic, Socio-Cultural, and Technological.

\subsubsection{Political}
Governments are likely to impose more regulations around data collections, privacy concerns and possibly on data transfers between countries e.g., U.S. Export control.
An example that illustrates how we continuously adapt to legislation is the data collection and analysis around Service work utilization. We cannot drill down in the analysis to the lowest level as that would link time utilization to individual FSEs.

The risk of political instability in markets is mitigated by subscribing to external partners that scans the news-feeds for early warning signs. Already today the company subscribes to a Business Travel Safety service from an external company that most likely scans live news-feeds from for example Reuters. In the past companies scanned newspapers for news about political deregulation in their business field. Changes in regulations in the food industry that creates new business opportunities will continue. However, today such opportunities are often found using data analytics on live news data.

\subsection{Economic}
There is a continuous shift in many businesses towards becoming more data-driven. Companies that understand the value that is generated in their own internal data bases combined with external data sets will have a competitive advantage. The rest will have to accept being acted upon in the market rather than being in control of the first-mover initiative.

\subsection{Socio-Cultural}
Data-driven companies understand the value in collecting data, internal data as well as external data from customers and consumers. Google is a good example of a company that both collects data and give it back to consumers as value added data. The benefit for the consumer is for example a better dining experience as you can tap into restaurant reviews in google maps. As long as privacy concerns are managed, more consumers tend to accept the sharing and collection of data.

Likewise, companies in B2B transactions become increasingly aware of the risks but also the benefits in sharing and collecting data if they see the return in added value services. In our business we have the example of TPMS OnLine where the collection of running hours and service event logging, generates work orders through an Asset Care Center. The customers benefit from higher Equipment availability provided that they share information and that Tetra Pak returns value added services. 

In the future customers will, by being influenced by the Socio-Cultural environment, not only accept data utilization, but demand that they receive data-driven services in the entire value chain. Marketing, Production process optimizations, and Asset Care are areas in the value chain that hold opportunities for such value added transactions.

As consumers we also expect innovative new products and services through recombining data and technology applications, as for example self-driving cars. Industrial customers expect similar fusions of creativity and technology applications to take place.

\subsection{Technological}<++>
%------------------------------------------------
\section{Strategy}
text

\subsection{Scope}

text

\subsection{Goals and Objectives}

text

\subsection{Resource deployments}

text

\subsection{Sustainable competitive advantage}

text

\subsection{Synergies}

text


%------------------------------------------------
\section{Execution}

\subsection{subsection}

text


\subsection{Developing a Data-and-Analytics Plan}

To be successful in the organization, the investment in building up a data driven TS\&S business must start delivering results in 6-9 months. This will keep momentum going and ensure continued organizational support and focus. The first step is to decide what the outcome should be, e.g., increase sales of pistons by \textit{x} \% in market \textit{y} after \textit{z} months and thereby breaking a declining sales trend. The ROI is now quantifiable and a business case can be developed. It is important to set goals\footnote{SMART, i.e, Specific, Measurable, Achievable, Resource based, and Time bound} before starting data mining activities ad hoc. With a set of goals the strategy to get there can be developed. Obviously, deep dives into data will not in itself increase sales in the example above. However, data driven decisions that correctly allocate scarce sales resources will. This also implies that the sales organization in the example currently have an insufficient decision making model for resource deployment, and is open for changing to data driven decision models.


\subsection{Roadmap}

text

\subsection{Data access}

text

\subsection{S/W Tools}

text

\subsection{HR Profiles}

text


%------------------------------------------------
\section{Summary}

text

%-----------------------------------------------
\section{Recommendations}

text
%------------------------------------------------

\end{document}
